
%!TEX spellcheck = en-US
%%%%%%%%%%%%%%%%%%%%%%%%%%%%%%%%%%%%%%%%%%%%%%%%%%%%%%%%%%%%%%%%%%%%%%%%%%%%
%% Author template for Management Science (mnsc) for articles with no e-companion (EC)
%% Mirko Janc, Ph.D., INFORMS, mirko.janc@informs.org
%% ver. 0.95, December 2010
%%%%%%%%%%%%%%%%%%%%%%%%%%%%%%%%%%%%%%%%%%%%%%%%%%%%%%%%%%%%%%%%%%%%%%%%%%%%
\documentclass[ms]{style/informs3}%,blindrev
% \documentclass[mnsc,blindrev]{style/informsX}%,blindrev
%\documentclass[mnsc,nonblindrev]{informs3} % current default for manuscript submission

\OneAndAHalfSpacedXI

\usepackage{natbib}
 \bibpunct[, ]{(}{)}{,}{a}{}{,}%
 \def\bibfont{\small}%
 \def\bibsep{\smallskipamount}%
 \def\bibhang{24pt}%
 \def\newblock{\ }%
 \def\BIBand{and}%

\setlength{\bibsep}{-1pt}

\TheoremsNumberedThrough     % Preferred (Theorem 1, Lemma 1, Theorem 2)
\ECRepeatTheorems
\EquationsNumberedThrough    % Default: (1), (2), ...

\MANUSCRIPTNO{}

%%%%%%%%%%%%%
%SPECIAL PACKAGES and COMMANDS
%%%%%%%%%%%%%
\usepackage{booktabs}
\usepackage{bbm}
\usepackage{bm}
\usepackage[autostyle]{csquotes}
\usepackage{tabularx}
\usepackage{multirow}
\usepackage{dcolumn}
\usepackage{hyperref}
\usepackage{makecell}
\usepackage{subcaption}
\usepackage{mathrsfs,amsmath}
\usepackage{algorithm} %[plain]
\usepackage[noend]{algorithmic}
\newlength\myindent
\setlength\myindent{2em}
\newcommand\bindent{%
  \begingroup
  \setlength{\itemindent}{\myindent}
  \addtolength{\algorithmicindent}{\myindent}
}
\newcommand\eindent{\endgroup}

% \usepackage{amsmath}
% \newcommand\numberthis{\addtocounter{equation}{1}\tag{\theequation}}

\makeatletter
\def\munderbar#1{\underline{\sbox\tw@{$#1$}\dp\tw@\z@\box\tw@}}
\makeatother

\usepackage{tikz}
% \usepackage{graphicx}
\usepackage[all]{xy}
\usetikzlibrary{decorations.pathmorphing} % noisy shapes
\usetikzlibrary{fit}					% fitting shapes to coordinates
\usetikzlibrary{backgrounds}	% drawing the background after the foreground
\usetikzlibrary{tikzmark,decorations.pathreplacing,calligraphy}

\usepackage{pgfplots}
\pgfplotsset{width=15cm,height=8cm}
\usepgfplotslibrary{fillbetween}

%%%> colors
\definecolor{myblue}{RGB}{51,51,178}
\definecolor{myred}{RGB}{178,51,51}
\definecolor{myblack}{RGB}{0,0,0}
\definecolor{mygreen}{RGB}{0,128,0}
\definecolor{mygray}{gray}{0.6}

%\newcommand{\dani}[1]{{\color{red}\textbf{[D:#1]}}}
\newcommand{\nacho}[1]{{\color{mygreen}[#1]}}
\newcommand{\av}[1]{{\color{myblue}[#1]}}

% delimiters
\usepackage{mathtools}
\DeclarePairedDelimiter{\ceil}{\lceil}{\rceil}
\newcommand{\lrp}[1]{\left(#1\right)}
\newcommand{\lrcp}[1]{\left[#1\right)}
\newcommand{\lrpc}[1]{\left(#1\right]}
\newcommand{\lrc}[1]{\left[#1\right]}
\newcommand{\lrl}[1]{\left\{#1\right\}}
\newcommand{\lra}[1]{\left|#1\right|}
\newcommand{\lrn}[1]{\left\lVert#1\right\rVert}
\newcommand{\lran}[1]{\langle #1 \rangle}

\newcommand{\R}{\mathbb{R}}
\newcommand{\N}{\mathbb{N}}
\newcommand{\E}{\mathbb{E}}


\newcommand{\ind}[1]{\mathbbm{1}_{\lrl{#1}}}
\newcommand{\indicator}[1]{\mathbbm{1}\lrl{#1}}

\newcommand{\prob}[1]{\textup{P}\lrp{#1}}
\newcommand{\piecewise}[1]{\left\{\begin{array}{ll} #1 \end{array}\right.}


\newcommand{\cA}{\mathcal{A}}
\newcommand{\cB}{\mathcal{B}}
\newcommand{\cC}{\mathcal{C}}
\newcommand{\cD}{\mathcal{D}}
\newcommand{\cI}{\mathcal{I}}
\newcommand{\cJ}{\mathcal{J}}
\newcommand{\cK}{\mathcal{K}}
\newcommand{\cH}{\mathcal{H}}
\newcommand{\cN}{\mathcal{N}}
\newcommand{\cM}{\mathcal{M}}
\newcommand{\cT}{\mathcal{T}}
\newcommand{\cX}{\mathcal{X}}
\newcommand{\cY}{\mathcal{Y}}
\newcommand{\cZ}{\mathcal{Z}}
\newcommand{\cS}{\mathcal{S}}
\newcommand{\cP}{\mathcal{P}}
\newcommand{\cQ}{\mathcal{Q}}
\newcommand{\cR}{\mathcal{R}}
\newcommand{\cF}{\mathcal{F}}
\newcommand{\cG}{\mathcal{G}}
\newcommand{\cV}{\mathcal{V}}
\newcommand{\cL}{\mathcal{L}}

\newcommand{\bA}{\bar{A}}
\newcommand{\bB}{\bar{B}}
\newcommand{\bC}{\bar{C}}
\newcommand{\bD}{\bar{D}}
\newcommand{\bM}{\bar{M}}
\newcommand{\bT}{\bar{T}}
\newcommand{\bX}{\bar{X}}
\newcommand{\bS}{\bar{S}}
\newcommand{\bP}{\bar{P}}
\newcommand{\bQ}{\bar{Q}}
\newcommand{\bR}{\bar{R}}

\newcommand{\bdC}{\mathbf{C}}
\newcommand{\bdS}{\mathbf{S}}
\newcommand{\bdR}{\mathbf{R}}

\newcommand{\bcK}{\overline{\mathcal{K}}}

\newcommand{\tA}{\tilde{A}}
\newcommand{\tb}{\tilde{b}}
\newcommand{\tB}{\tilde{B}}


\newcommand{\sym}[1]{{#1}}


\newtheorem{result}{Result}[section]

%%%%%%%%%%%%%%%%
%%%%NOTATION%%%%
%%%%%%%%%%%%%%%%


\usepackage{url}
\newcommand{\update}[1]{{\color{myred}\textbf{#1}}}
\newcommand{\del}[1]{{\color{green}{delete: #1}}}



%%%%%%%%%%%%%%%%
\begin{document}
%%%%%%%%%%%%%%%%

% Outcomment only when entries are known. Otherwise leave as is and
%   default values will be used.
%\setcounter{page}{1}
%\VOLUME{00}%
%\NO{0}%
%\MONTH{Xxxxx}% (month or a similar seasonal id)
%\YEAR{0000}% e.g., 2005
%\FIRSTPAGE{000}%
%\LASTPAGE{000}%
%\SHORTYEAR{00}% shortened year (two-digit)
%\ISSUE{0000} %
%\LONGFIRSTPAGE{0001} %
%\DOI{10.1287/xxxx.0000.0000}%

\RUNTITLE{Stable Matching with Dynamic Priorities}

\TITLE{Stable Matching with Dynamic Priorities}

\ARTICLEAUTHORS{%
\AUTHOR{Ignacio Rios}
\AFF{Naveen Jindal School of Management, University of Texas at Dallas, \EMAIL{ignacio.riosuribe@utdallas.edu} } %,\URL{https://iriosu.github.io/} }
%\EMAIL{iriosu@stanford.edu} %, \URL{}}
% Enter all authors
%\vspace*{0.5cm}
%\AUTHOR{{\large Preliminary and Incomplete: please do not circulate.}}
} % end of the block

\ABSTRACT{
  In many matching markets, agents may get prioritized depending on the assignment of other agents. Examples include (i) school choice, where students get prioritized if one of their siblings is assigned to a school; (ii) refugee resettlement, where the clearinghouse prioritizes the assignment of refugees with similar backgrounds (e.g., language) to the same cities; and (iii) the hospital-resident problem, where members of a couple may get prioritized depending on the assignment of their partner. In all these cases, priorities are dynamic and depend on the allocation. This paper introduces the problem of finding a student-optimal stable matching under dynamic priorities. We show that the problem is NP-hard, and we provide two integer programming formulations for the problem. Given the potentially high computational time to solve these formulations, we introduce several heuristics to preprocess the instances and solve them. Finally, we show that clearinghouses can significantly improve students' welfare when considering dynamic priorities using synthetic and real data from the Chilean school choice system.
}
%
\KEYWORDS{stable matching, many-to-one matching, school choice, integer programming.}



% ======================
% MAIN PAPER
% ======================
\maketitle

%!TEX root = ./0_main.tex
\section{Introduction}


The theory of two-sided many-to-one matching markets, introduced by~\cite{gale1962college}, provides a framework for solving many large-scale real-life assignment problems. Examples include entry-level labor markets for doctors and teachers, education markets from daycare, school choice and college admissions, and other applications such as refugee resettlement. A common feature in many of these markets is the use of mechanisms that find a stable assignment, as this guarantees that no coalition of agents has incentives to circumvent the match.

In many of these markets, the clearinghouse may be interested in finding a stable allocation, and individual agents care about their assignment and that of other agents. In the hospital-resident problem, couples jointly participate and must coordinate to find two positions that complement each other. In school choice, students may prefer to be assigned with their siblings or neighbors. In refugee resettlement, agencies may prioritize allocating families with similar backgrounds (e.g., from the same region or speaking the same language) to the same cities. One approach to accommodate these joint preferences is to provide priorities contingent on the assignment. For example, many school choice systems (including NYC, New Haven, Denver, Chile, etc.) consider \emph{sibling priorities}, by which students get prioritized in schools where they have a sibling currently assigned or enrolled. In refugee resettlement, families may get higher priority in localities where they have relatives based on \emph{family reunification}. As a result, priorities may not be fixed and may depend on the current assignment. We refer to these as \emph{dynamic priorities}.

In this paper, we study the problem of finding a stable matching under \emph{dynamic priorities}, using school choice with siblings as a motivating example. To accomplish this, we first introduce a stylized model where students belong to (potentially different) grade levels and get higher priority in a given school if they have a sibling assigned to it. We show that approaches used in practice, such as sequentially solving the assignment for each grade level and updating priorities, can lead to suboptimal results. Moreover, we show that the problem of finding the maximum cardinality stable assignment under \emph{dynamic priorities} is NP-hard.
Given these results, we focus on deriving exact and near-optimal approaches to solve the problem. Specifically, we provide two exact formulations of the problem, and we introduce several heuristics to efficiently obtain near-optimal solutions. Finally, using data from the Chilean school choice system, we show that our framework can significantly increase the welfare of students by assigning more students with their siblings and increasing the overall number of students assigned.

\subsection{Contributions}
Our paper contributes to the literature in many dimensions:

\paragraph{Problem formulation.} In this area:
\begin{itemize}
  \item Identify issues with current practice: arbitrary preferences and priorities impact number of students assigned.
  \item Propose a model that captures main elements problems in practice, including dynamic priorities (e.g., siblings).
  \item Show that problem is hard.
\end{itemize}

\paragraph{Stability concept.} In this area:
\begin{itemize}
  \item Introduce two notions of stability: dependent and independent.
  \item Show existence
  \item Strategy-proofness
\end{itemize}

\paragraph{Mathematical programming approach.} In this area:
\begin{itemize}
  \item Propose integer programming formulation to address the problem.
  \item Show correctness
  \item Show that different goals (maximal use of reservations, maximize size of match) can be achieved by modifying the objective.
  \item Propose algorithm to reduce size of instance
  \item Heuristics
\end{itemize}

\paragraph{Application and impact.} In this area:
\begin{itemize}
  \item Apply our framework to Chile.
  \item Show that the welfare of students can be significantly increase with our approach.
\end{itemize}

The remainder of this paper is organized as follows. In Section~\ref{sec: literature} we discuss the relevant literature. In Section~\ref{sec: model} we introduce our model. In Sections~\ref{sec: exact} we study exact formulations of the problem. In Section~\ref{sec: heuristics} we describe heuristics to reduce the size of the problem and obtain near-optimal solutions in a relatively short time. In Section~\ref{sec: experiments} we adapt our framework to the Chilean school choice problem and show its potential benefits compared to the current practice. Finally, in Section~\ref{sec: conclusions} we conclude.


%!TEX root = ./0_main.tex
\section{Literature}\label{sec: literature}

Our paper is related to several strands of the literature.

\paragraph{School choice.} Starting with \cite{Abdulkadiroglu2003}, a large body of literature has studied different elements of the school choice problem, including the use of different mechanisms such as DA, Boston, and TTC~\citep{abdulkadirouglu2005boston}; the use of different tie-breaking rules~\citep{Arnosti_2015,Ashlagi_2019}; the handling of multiple and potentially overlapping quotas; the addition of affirmative action policies; and the implementation in many school districts and countries.

 \cite{Abdulkadiroglu2003}, \citep{abdulkadirouglu2005new}, \citep{abdulkadirouglu2005boston}.  Barcelona \citep{calsamiglia2014illusion}, Amsterdam \citep{gautier2016eerste} and Chile~\cite{Correa_2022} have implemented centralized school choice systems using some variant of DA, BM, or Top-Trading Cycles (TTC). \cite{Abdulkadiroglu2003} also initiated a large literature that theoretically analyzes the school choice problem.

\paragraph{Optimization in stable matching.}
\begin{itemize}
  \item Starters: \cite{gusfield1989stable, vate1989linear,rothblum1992characterization,roth1993stable}
  \item Formulations: \cite{baiou2000stable}, \cite{kwanashie2014integer}, \cite{agoston2016integer}, \cite{agoston_etal2021}, \cite{delorme2019mathematical}.
  \item Alternative objectives: \cite{shi2016}, \cite{Ashlagi_2016}, \cite{bodon2020}, \cite{bobbio22}
\end{itemize}


\paragraph{Siblings.} \cite{Dur_2022}, \cite{Dur_2019}, \cite{Correa_2022}

\paragraph{Complementarities.}


%!TEX root = ./0_main.tex

\section{Model}\label{sec: model}

    We formalize the \emph{stable matching problem with dynamic priorities} using school choice with siblings as a motivating example.

    Let \(\cS\) be the set of students, and let \(\cF\) be a partition of students into families, so that \(f(s) \in \cF\) represents the family of student \(s\in \cS\). Then, we say that students \(s\) and \(s'\) are siblings if \(f(s) = f(s')\), and we say that \(s\) has no siblings if \(f(s) = \lrl{s}\).  In addition, we assume that each family \(f\) has a strict preference order \(\succ_f\) over tuples in the set \(\lrp{C\cup\lrl{\emptyset}}^{\lra{f}}\), where \(\emptyset\) represents that a student is unassigned. Specifically, if \(f = (f_1, f_2)\), then \(c,c' \succ_f c'', c'''\) implies that family \(f\) prefers that \(f_1\) and \(f_2\) go to schools \(c\) and \(c'\) over \(c''\) and \(c'''\), respectively.\footnote{In a slight abuse of notation, we use \(\sim_f\) to refer that family \(f\) is indifferent between two subsets of schools, i.e., for any \(\vec{c}, \vec{c}' \in \lrp{\cC \cup \lrl{\emptyset}}^{\lra{f}}\), \(\vec{c} \sim_f \vec{c}'\) if neither \(\vec{c}\succ_f \vec{c}'\) nor \(\vec{c}'\succ_f \vec{c}\) hold.}
    Finally, let \(\cG\) be the set of grade levels, \(g(s)\) the grade of student \(s\), \(\cS^{g} \subseteq \cS\) the set of students applying to grade level \(g\in \cG\), and \(S^g_c\) the set of students applying to school \(c\in \cC\) in grade \(g\in \cG\).\footnote{Notice that the model applies directly in cases where there is a ``single grade level'', as in refugee resettlement, college admissions and in the hospital-resident problem.}

    On the other side of the market, let \(\cC\) be the set of schools. Each school \(c\) offers \(q_{cg}\) seats on each grade \(g\in \cG\), with \(q_{cg} = 0\) representing that school \(c\) does not offer grade \(g\). For simplicity, we assume that there are no special reserves or quotas.\footnote{In the Appendix we extend our formulation to include overlapping types and reserves.} In addition, we assume that all schools consider a \emph{sibling priority}, i.e., they prioritize students who have a sibling assigned or enrolled in the school. As a result, each school has two priority groups: (i) students with siblings and (ii) students with no siblings in the school, where students in the former group are strictly preferred to students in the latter. Within each priority group, schools break ties with a random tie-breaker. It is important to notice that the random tie-breaker defines a strict order of students only within each priority group; however, it does not define an order among any two pair of students, as they may change from one priority group to the other depending on the assignment of their siblings. Nevertheless, we assume that each school \(c\) has preferences over sets of students in \(2^{\cS}\), which we denote by \(\succ_c\).

    Let \(\cV \subseteq \cS \times \cC\cup \lrl{\emptyset}\) be the seat of feasible pairs, i.e., \((s,c) \in \cV\) implies that student \(s\) applied to \(c\), satisfies all the admission requirements, and \(q_{c,g(s)} > 0\). Similarly, let \(\cV^{g} \subseteq \cV\) be the set of feasible pairs including students applying to grade \(g\in \cG\). A matching is an assignment \(\mu\subseteq \cV\) such that (i) each student is assigned to at most one school in \(\cC\), and (ii) each school is assigned at most its capacity in each grade level. Formally, given an assignment \(\mu\), let \(\mu(s) \in \cC\cup \lrl{\emptyset}\) be the school assigned to student \(s\), and let \(\mu(c)\) be the set of students assigned to school \(c\). Then, a matching satisfies that (i) \(\mu(s) \in \cC\cup \lrl{\emptyset}\) for all students \(s\in \cS\) and (ii) \(\lra{\lrl{s \in \mu(c)\;:\; g(s) = g}} \leq q_{cg}\) for all schools \(c\in \cC\) and grade levels \(g\in \cG\).

    As \cite{roth02} discusses, a desired property of any assignment is stability, i.e., that there are no pair of agents that prefer to circumvent the match and be assigned to each other. Given an assignment \(\mu\), we say that student \(s\) has \emph{justified envy} towards another student \(s'\) assigned to school \(c\) if (i) \(g(s)=g(s')\), (ii) \(\lrp{c, \mu(f\setminus \lrl{s})} \succ_f \mu(f)\), and(iii) \(\mu(c) \cup \lrl{s} \setminus \lrl{s'} \succ_c \mu(c)\). In words, the first condition states that both students belong to the same grade level; the second condition implies that the family prefer that \(s\in f\) is assigned to \(c\) given the assignment of their siblings; and the third condition states that school \(c\) prefers the set of student that replaces \(s'\) with \(s\). In addition, we say that an assignment \(\mu\) is \emph{non-wasteful} if there is no student \(s\in \cS\) and school \(c\) such that \(\mu(c, \mu\lrp{f\setminus\lrl{s}}) \succ_f \mu(f)\) and \(\lra{ \lrl{ s'\in \mu(c)\;:\; g(s') = g(s)} } < q_{cg}\). Then, we say that an assignment is \emph{stable} if it is \emph{envy-free} (i.e., no student has \emph{justified envy}) and \emph{non-wasteful}.

    As~\cite{Correa_2022} discuss, a stable assignment may not exist (see Theorem~\ref{thm: no stable matching} in Appendix~\ref{app: additional results}). Moreover, families may not be able to report preferences over tuples of schools, as the number of combinations increases exponentially in the number of siblings applying to the system. For these reasons, many school systems elicit a preference list for each family member, i.e., a set of preference orders \(\lrl{\succ_i}_{i\in f}\) among schools in \(\cC\cup \lrl{\emptyset}\). Then, as proposed in~\cite{Correa_2022}, one option to solve the problem is to define an order in which grades are processed and sequentially solve the assignment of each grade level using the student-optimal variant of DA. More specifically, the algorithm in~\cite{Correa_2022} starts processing the highest grade (i.e., 12th grade). Then, before moving to the next grade, the sibling priorities are updated, considering the assignment of the grade levels already processed. After processing the final grade level (i.e., Pre-K), this procedure finishes.

    Notice that the heuristic mentioned above obtains a stable assignment is the preferences of families satisfy \emph{higher-first}, i.e., each family prioritizes the assignment of their oldest member (see Proposition 2 in~\cite{Correa_2022}). However, this is not the case if some families' preferences do not satisfy this condition. In addition, as Example~\ref{ex: order of grades matters} illustrates, the order in which grades are processed matters.
    \begin{example}\label{ex: order of grades matters}
        Consider an instance with two grades \(g_1 < g_2\), one family \(f = \lrl{f_1,f_2}\), and two additional students \(\lrl{a_1, a_2}\). Students \(f_1\) and \(a_1\) apply to grade \(g_1\),
        and \(f_2\) and \(a_2\) apply to grade \(g_2\). Finally, the preferences and priorities are:
        \begin{equation}
          \begin{aligned}
            \succ_{f_1}&: c_2, c_1, \quad &\succ_{a_1}: c_2, c_1   \\
            \succ_{f_2}&: c_1, c_2, \quad &\succ_{a_2}: c_1, c_2   \\
            \succ_{c_1}^{g_1}&: a_1, f_1, \quad &\succ_{c_1}^{g_2}: a_2, f_2 \\
            \succ_{c_2}^{g_1}&: a_1, f_1, \quad &\succ_{c_2}^{g_2}: a_2, f_2 \\
          \end{aligned}
        \end{equation}
        We observe that, if grades are processed in decreasing order (as in Chile), we obtain the assignment \(\mu = \lrl{(f_1, c_1), (a_1, c_2), (f_2, c_1), (a_2, c_2)}\). In contrast, if we process grades in increasing order, we obtain the assignment \(\mu' = \lrl{(f_1, c_2), (a_1, c_1), (f_2, c_2), (a_2, c_1)}\).
        \hfill \(\square\)
    \end{example}

    Moreover, defining an order in which grades must be processed imposes an additional constraint that may rule out a stable assignment that is weakly preferred by every family (and strictly preferred by at least one family). This is illustrated in Example~\ref{ex: flexibility helps welfare}.
    \begin{example}\label{ex: flexibility helps welfare}
        Consider an instance with two grades, \(G = \lrl{g_1, g_2}, \; g_1 < g_2\), two families \(f = \lrl{f_1, f_2}, f' = \lrl{f'_1, f'_2}\), and a set of students \(S = \lrl{f_1, f_2, f_1', f_2', a_1, a_2, a_3, a_4}\).
        In addition, consider two schools \(C = \lrl{c_1, c_2}\) offering two seats in each grade. Finally, suppose that the preferences of students are given by:
        \begin{equation}
          \begin{aligned}
            f&: (c_1, c_1) \sim (c_2, c_2) \succ (c_1, c_2) \succ (c_2, c_1) ,  & \quad a_1: c_1 \\
            f'&: (c_1, c_1) \sim (c_2, c_2) \succ (c_1, c_2) \succ (c_2, c_1),  & \quad a_2: c_2 \\
            i&: c_1 \succ c_2, \; \forall i \in \lrl{f_1, f_1'}, & \quad a_3: c_1 \\
            i&: c_2 \succ c_1, \; \forall i \in \lrl{f_2, f_2'},  & \quad a_4: c_2, \\
          \end{aligned}
        \end{equation}\
        while schools priorities are:
        \begin{equation}
          \begin{aligned}
            c_1&: f \succ f' \succ \lrl{a_1,a_3} \\
            c_2&: f \succ f' \succ \lrl{a_2,a_4}
          \end{aligned}
        \end{equation}
        If we process grades starting with \(g_1\), we obtain the assignment
        \[\mu = \lrl{(f_1,c_1), (f_2,c_1), (f'_1,c_1), (f'_2,c_1), (a_1, \emptyset), (a_2, c_2), (a_3, \emptyset), (a_4, c_2)},\]
        i.e., two students result unassigned. However, notice that the allocation
        \[\mu' = \lrl{(f_1,c_1), (f_2,c_1), (f'_1,c_2), (f'_2,c_2), (a_1, c_1), (a_2, c_2), (a_3, c_1), (a_4, c_2)}\]
        satisfies our notion (to be refined) of stability and leads to no students unassigned.
        \hfill \(\square\)
    \end{example}

    Examples\ref{ex: order of grades matters} and~\ref{ex: flexibility helps welfare} show that the order in which grade levels are processed matters and that imposing an order may reduce the total number of students assigned. These two examples motivate Problem~\ref{problem_def}, which aims to find the student-optimal stable assignment of maximum cardinality by leveraging the siblings' priority.


\subsection{Practical preferences and priorities}
    The definition of \emph{justified envy} in the previous section assumes that schools have preferences over sets of students, and that families have joint preferences over sets of schools. However, in most clearinghouses, preferences and priorities are not as complex, and simply involve preferences by students and a combination of random tie-breakers and priority groups to define schools' priorities. For this reason, in the remainder of the paper we will assume a simplified structure of preferences and priorities, as formalized in Assumption~\ref{assump: simplified preferences and priorities}.

    \begin{assumption}\label{assump: simplified preferences and priorities}
      We assume that students' preferences and school priorities are such that:
      \begin{enumerate}
        \item On the students side, we assume that each student report a strict preference list.
        \item On the schools side, we assume that each school ranks students according to a random tie-breaker, and that there is a single priority group that gives higher priority to students with siblings assigned or enrolled in the school. %Moreover, if two students have siblings priority, they are order according to their own random tie-breaker.
      \end{enumerate}
    \end{assumption}
    Given this assumption, we will use \(\succ_c\) to represent ex-ante priorities at school \(c\), i.e., the priorities derived from the random tie-breaker without considering siblings priorities.
    Although Assumption~\ref{assump: simplified preferences and priorities} simplifies the reporting of preferences, the siblings' priority needs some limitations to ensure the fairness of the assignment, as the following example illustrates.

    \begin{example}\label{ex: unlawful siblings priority}
      Consider an instance with a single level, a set of students \(\cS = \lrl{a_1, a_2, f_1, f_2}\) where \(f_1\) and \(f_2\) are siblings, and a single school with two seats. Moreover, suppose ex-ante priorities are \(a_1 \succ_c a_2 \succ_c f_1 \succ_c f_2\). Then, one possible assignment is \(\mu = \lrl{(a_1, c), (a_2, c), (f_1, \emptyset), (f_2, \emptyset)}\). However, the alternative assignment \(\mu' = \lrl{(a_1, \emptyset), (a_2, \emptyset), (f_1, c), (f_2,c)}\) would still be feasible, as both \(f_1\) and \(f_2\) have siblings' priority and thus have the highest priority ex-post. \hfill \(\square\)
    \end{example}

    We claim that assignment \(\mu'\) in Example~\ref{ex: unlawful siblings priority} is not desirable, since neither \(f_1\) nor \(f_2\) would be admitted without siblings priority. To rule out this possibility, we restrict attention to assignments that satisfy \emph{unilateral dynamic priorities}, which are formalized in Definition~\ref{def: unilateral priorities}.
    \begin{definition}\label{def: unilateral priorities}
      Consider an assignment \(\mu \subseteq \cV\). Then, \(\mu\) satisfies \emph{unilateral dynamic priorities} if, for any student \(s'\in f\) with a sibling \(s''\in f\) assigned to the same school \(c\), either \(s'\in\mu_{-s''}(c)\) or \(s''\in \mu_{-s'}(c)\), where \(\mu_{-s}\) is the assignment that would be obtained if student \(s\) were excluded from the set of students.
    \end{definition}
    In words, \emph{unilateral priorities} ensure that at least one of the siblings is independently assigned to the school, potentially providing priority to their other siblings.

    It remains to describe how ties are broken among students with siblings priority. One possible option, which we refer to as the \emph{dependent rule}, is that students that ties among students that get prioritized are broken based on the priority of their family member that granted them their siblings priority. The other alternative, which we refer to as the \emph{independent rule}, is that we break ties among prioritized students based on their own original priority. In Example~\ref{ex: breaking ties among siblings} we illustrate these two rules.

    \begin{example}\label{ex: breaking ties among siblings}
      Consider a single school \(c\) offering three seats, and two families, \(f = \lrl{f_1, f_2}, g=\lrl{g_1,g_2}\), with all students applying to the same grade. Moreover, suppose that ex-ante priorities are \(f_1 \succ_c g_1 \succ_c g_2 \succ_c f_2\). Then, \(\mu(f_1)=c\) and \(\mu(g_1)=c\). As a result, both \(f_2\) and \(g_2\) get siblings priority, but there is only one seat left. If the \emph{dependent rule} is in place, then \(\mu(f_2) = c\) and \(\mu(g_2) = \emptyset\), since \(f_1 \succ_c g_1\). On the other hand, if the \emph{independent rule} is in place, \(\mu(f_2) = \emptyset\) and \(\mu(g_2) = c\), since \(g_2 \succ_c f_2\).
    \end{example}

    Note that the \emph{dependent} and the \emph{independent} rules are used in practice. On the one hand, the former is used in Chile \citep{Correa_2022}, as the clearinghouse breaks ties at the family level first, and then breaks ties within each family. On the other hand, the latter is used in NYC to break ties among students with siblings priority \nacho{Confirm that this is the case.}. Hence, which rule to use is a policy relevant question that we evaluate in Section~\ref{sec: experiments}.

    Based on these two rules (i.e., \emph{dependent vs. independent rules}), we define two notions of justified-envy: (i) \emph{dependent-justified envy} and (ii) \emph{independent-justified envy}. Before formally describing these, let \(\zeta_{\mu}(s,c) = \argmax_{k\in f(s)\setminus \lrl{s}}\lrl{p_{k,c} \;:\; \mu(k) = c, \; k\succ_c s }\) be the function that returns the sibling of student \(s\) assigned to \(c\) with the highest ex-ante priority. In a slight abuse of notation, we use \(\zeta_{\mu}(s,c) = \emptyset\) if there is no such sibling, and thus \(\zeta_{\mu}(s,c) \sim_c \zeta_{\mu}(s',c)\) if and only if both \(s\) and \(s'\) have no siblings assigned to school \(c\).

    \begin{definition}{(Dependent-justified envy)}
      A student \(s\) has dependent-justified envy toward another student \(s'\) assigned to a school \(c\) if (i) \(g(s) = g(s')\), (ii) \(c \succ_s \mu(s)\), and (iii) either \(\zeta_{\mu}(s,c) \succ_c \zeta_{\mu}(s',c)\) or \(\zeta_{\mu}(s,c) \sim_c \zeta_{\mu}(s',c)\) and \(s\succ_c s'\).
    \end{definition}

    \begin{definition}{(Independent-justified envy)}
      A student \(s\) has independent-justified envy toward another student \(s'\) assigned to a school \(c\) if (i) \(g(s) = g(s')\), (ii) \(c \succ_s \mu(s)\), and (iii) either \(\lra{\zeta_{\mu}(s,c)} > \lra{\zeta_{\mu}(s',c)}\) or \(\lra{\zeta_{\mu}(s,c)} = \lra{\zeta_{\mu}(s',c)}\) and \(s\succ_c s'\).
    \end{definition}

    % \begin{example}
    %   Consider an instance with a single family \(f=\lrl{f_1, f_2}\), a set of students \(\cS = \lrl{s, f_1, f_2}\), and a single school \(c\) with capacity \(q_c=2\). Every student prefer \(c\) to being unassigned, while ex-ante priorities at school \(c\) are: \(\succ_c: s \succ f_1 \succ f_2\).
    %   Notice that there are two possible non-wasteful assignments:
    %   \begin{enumerate}
    %     \item Relative priority: \(\mu = \lrl{(s,c), (f_1, c), (f_2, \emptyset)}\)
    %     \item Absolute priority: \(\mu = \lrl{(s,\emptyset), (f_1, c), (f_2, c)}\)
    %   \end{enumerate}
    %   \hfill \(\square\)
    % \end{example}
    %
    % In the reminder of this paper we focus on relative priorities. In the Appendix we extend our formulations to the case with absolute priorities, and we show that the total number of siblings assigned to the same schools is larger for the case with absolute priorities.
    Finally, we say that an allocation is (in)dependent-stable if it is (in)dependent-justify envy free and non-wasteful.

\subsection{Maximum cardinality stable matching}
    We now formalize the problem that we focus on this paper.
    \begin{problem}\label{problem_def}
    Given an instance $\Gamma$, the \emph{stable-matching with unilateral dynamic priorities} aims to find a student-optimal stable-assignment with unilateral priorities of maximum cardinality.
    % \begin{equation*}
    % \min_{\mu} \Bigg\{ \sum_{(s,c)\in \mu} r_{s,c} \; : \;  \mu \text{ is a stable matching }
    %  \Bigg\}.
    % \end{equation*}
    \end{problem}
    Problem~\ref{problem_def} is relevant from a policy perspective because it combines two of the primary goals in many matching markets: (i) to find an assignment involving as many agents as possible, and (ii) incorporate dynamic priorities. However, in Theorem~\ref{thm: problem is np hard} we show that Problem~\ref{problem_def} is NP-Hard.

    \begin{theorem}\label{thm: problem is np hard}
      Problem~\ref{problem_def} is NP-hard, independent of the rule used to break ties among students with siblings priority.
    \end{theorem}
    \begin{proof}{Proof.}
      \nacho{TODO: I guess we should follow Manlove et al. 2002}
      \nacho{I think the easiest way to prove this is by showing that the problem with siblings can be modeled as an instance of the problem with ties and incomplete lists, which is known to be NP-complete.
      Manlove 2002 shows that the problem is NP-complete even if ties are in only one side and ties are of length max 2; if we think of siblings as being tied, then this would exactly apply.}
      % We know that the maximum cardinality stable-matching problem with ties is NP-hard, even if ties are only on one side, and the tail of the preferences, and of size at most 2. \nacho{cite Manlove 2002}
      % We refer to this latter problem as SMTIR. We will show that we can reduce this problem to an instance of the stable matching problem with dynamic priorities.
      %
      % An instance of SMTIR consists of a set of students \(M\) with strict preferences, and a set of schools \(W\) whose lists of priorities may have ties at the tail of size at most 2. The decision problem is whether there exists a stable matching with size greater than or equal to \(K\).
      %
      % Given an instance of SMTIR, we construct an instance of SMD as follows. For each student \(i\in M\) in SMTIR we have a student \(i\in I\) in SMD. In addition, each time student \(i\) is in a tie in the priorities of a given school \(j\) (if any), we create a sibling of student \(i\) that applies to only that school \(j\) on a different grade \(g'\). Notice that a student \(i\in M\) may have multiple siblings in \(I\), since \(i\) will have a different sibling each time he is in a tie. Let \(S\) be the cardinality of the siblings added, and let \(r_{i',j} = 0\) for any sibling \(i'\) in any school \(j\).
      %
      % We now show that there exists solution to SMTIR generating at least \(K\) matches if and only if there exists a solution to SMD generating \(K\) matches.
      % \begin{itemize}
      %   \item[\(\Rightarrow\)] Suppose there exists a match \(\mu\) for SMTIR that generates at least \(K\) matches. Then, we construct a feasible match \(\mu'\) for SMD that generates at least \(K\) matches. First, we let \(\mu_{i}' = \mu_i\) for each \(i\in I\cap M\). We know that this is a feasible assignment in SMD since each student is assigned to one school, and each school is assigned at most its capacity. In addition, since \(r_{i,j} = 1\) for all \(i\in M\cap I\)and \(j\in J\), we know that \(\sum_{i,j\in \mu'}r_{i,j}\geq K\).
      %   It remains to show that the resulting match \(\mu'\) is \emph{ex-post} stable. To see this, \nacho{complete.}
      %   \item[\(\Leftarrow\)] Suppose that there exists a matching \(\mu'\) for SMD that generates a value of at least \(K\). Since the objective is greater than or equal to \(K\) and \(\mu'\) is a matchign and \(r_{i',j} = 0\) for all the artificial siblings, let \(\mu_i = \mu_i'\) for each \(i\in I\cap M\). It remains to show that \(\mu\) is a stable matching. \nacho{Compelte this}.
      % \end{itemize}
    \end{proof}

    \nacho{Maybe add some result regarding approximability? Maybe we can find something similar to what is in Bobbio et al. 2022.}


    % Given this hardness result, we will introduce exact formulations for the problem and new heuristics that will allow us to find near-optimal solutions in a reasonable time.


%!TEX root = ./0_main.tex
\section{Exact Formulations} \label{sec: exact}

    \nacho{I'm working on this now}

    Let \(x_{s,c}\) be a binary variable equal to 1 if the clearinghouse assigns student \(s \in \cS\) to school \(c \in \cC \cup \lrl{\emptyset}\), and 0 otherwise. Then, the set of matchings can be represented as
    \[
    \cP = \lrl{\mathbf{x} \in \lrl{0,1}^{\cV} \;:\; \text{for each } g\in \cG, \; \sum_{c: (s,c) \in \cV^{g}} x_{s,c} = 1 \text{ for all } s\in \cS, \; \sum_{s: (s,c) \in \cV^{g}} x_{s,c} \leq q_{c,g} \text{ for all } c\in \cC}.
    \]
    The first condition captures that student \(s\) is assigned to one school in the set \(\cC \cup \lrl{\emptyset}\), while the second condition implies that each school \(c\) receives at most \(q_{c,g}\) students.
    In addition, let \(r_{s,c}\) be the position of school \(c \in \cC\) in the preferences of student \(s \in \cS\), and let \(r_{s,\emptyset}\) be a parameter that captures the cost of having unassigned students. As \cite{agoston2016integer} and \cite{bobbio22} discuss, the problem of finding the student-optimal stable assignment in the standard setting (e.g., with no dynamic priorities) is equivalent to solving the following integer problem:
    \begin{subequations}\label{eq: standard formulation for student optimal stable matching}
    \begin{alignat}{2}
      \max_{\mathbf{x}\in \cP} \quad & \quad \sum_{(s,c)\in \cV} r_{s,c}\cdot x_{s,c} \label{eq: objective sosm} \\
      st. \quad & \quad q_{c,g(s)} \lrp{1- \sum_{c'\succeq_s c} x_{s,c'}} \leq \sum_{\substack{s' \in S^g(s):\\s'\succ_c s}} x_{s', c},\; \forall (s,c) \in \cV \label{eq: statibility for sosm}
    \end{alignat}
    \end{subequations}
    where the set of constraints~\ref{eq: statibility for sosm}, first introduced in~\cite{baiou2000stable}, ensure that the assignment is stable. Note that this problem is separable by grade level, and thus can be efficiently solved applying DA to each grade separately. However, as we will discuss next, considering dynamic priorities introduce dependencies between grade levels, and thus the problem is not separable anymore.

    \subsection{Base formulation}
        % Given a family \(f\in \cF\) with \(\lra{f} >1\), for any \(s,s'\in f\) let \(y_{s,s',c} :=   x_{s,c}\cdot x_{s',c}\). In words, \(y_{s,s',c}\) is equal to 1 if siblings \(s\) and \(s'\) are assigned to school \(c\), and 0 otherwise.
        % To linearly incorporate these variables into our formulation, we consider the McCormick expansion given by the set
        % \[
        % \cM(\mathbf{x}) = \lrl{\mathbf{y} \in [0,1]^{\cS\times \cV} \;:\; y_{s,s',c} \leq x_{s,c}, \; y_{s,s',c} \leq x_{s',c}, \; y_{s,s',c} \geq x_{s,c} + x_{s',c} - 1,\; s,s' \in f, \; f\in \cF, \; c\in \cC }.
        % \]
        Given a family \(f\in \cF\) with \(\lra{f} >1\), for any \(s,s'\in f\) with \(s\succ_c s'\), let \(y_{s,s',c}\) be a binary variable equal to 1 if \(s\) gives siblings' priority to \(s'\), and 0 otherwise. In words, \(y_{s,s',c}\) is equal to 1 if siblings \(s\) and \(s'\) are assigned to school \(c\) and \(s\) (i.e., \(x_{s,c} = x_{s',c}=1\)) and \(s\) has the highest priority among all siblings of \(s'\) assigned to school \(c\), and 0 otherwise.

        To linearly incorporate these variables into our formulation, we consider the set
        \[
        \begin{split}
        \cM(\mathbf{x}) &= \left\{ \mathbf{y} \in [0,1]^{\cS\times \cV} \;:\;  y_{s,s',c} \leq x_{s,c}, \; y_{s,s',c} \leq x_{s',c},
                                  \sum_{s''\in f(s'): s''\succ s'} y_{s'',s',c} \leq 1, \right. \\
          &\hspace{3cm} \left. \; \sum_{\substack{s''\in f:s'' \succeq_c s}} y_{s'',s',c} \geq x_{s,c} + x_{s',c} - 1,\; s,s' \in f \text{ with } s\succ_c s', \; f\in \cF, \; c\in \cC  \right\}
        \end{split}
        \]
        \nacho{Explain why this set correctly defines \(y\).}


        \subsubsection{Independent case.}
            Based on the set of feasible assignments \(\mathbf{x}\in \cP\) and the joint assignment variables \(\mathbf{y} \in \cM(\mathbf{x})\), we can formulate Problem~\ref{problem_def} by adapting~(\ref{eq: standard formulation for student optimal stable matching}) to incorporate the siblings priority.
            % \begin{subequations}\label{eq: standard formulation with siblings}
            % \begin{alignat}{2}
            %   \min_{\mathbf{x}\in \cP, \mathbf{y} \in \cM(\mathbf{x})} \quad & \quad \sum_{(s,c)\in \cV} r_{s,c}\cdot x_{s,c}  \\
            %   st. \quad & \quad q_{c,g(s)} \lrp{1- \sum_{c'\succeq_s c} x_{s,c'}} \leq \sum_{\substack{s'\in \cS^{g(s)}:\\ s'\succ_c s}} x_{s', c}
            %   + \sum_{\substack{(s',s'') \in \cS\times \cS^{g(s)} :\\ s'\succ_c s,\; s''\prec_c s }}  y_{s',s'',c}\; ,\; \forall (s,c) \in \cV, \; \lra{f(s)}=1 \label{eq: stability for no siblings} \\
            %   & \quad q_{c,g(s)} \lrp{1- \sum_{c'\succeq_s c} x_{s,c'}} \leq \sum_{\substack{s'\in \cS^{g(s)}:\\ s'\succ_c s}} x_{s', c} \;,\; \forall (s,c) \in \cV, \; \lra{f(s)} > 1  \label{eq: stability for siblings}
            % \end{alignat}
            % \end{subequations}
            % The sets of constraints~(\ref{eq: stability for no siblings}) and~(\ref{eq: stability for siblings}) adapt the stability constraints~(\ref{eq: statibility for sosm}) to incorporate dynamic priorities. Specifically,~(\ref{eq: stability for no siblings}) captures the case for students with no siblings. In that case, a student \(s\) may not get assigned to school \(c\) (or better) if there are at least \(q_{c,g(s)}\) students with (i) higher priority assigned to \(c\) (first term on the right-hand side) or (ii) lower priority than \(s\) but that have a sibling assigned to the school, and thus obtain siblings' priority.\nacho{Clarify the cases where the siblings is in the same grade or in a different one.} On the other hand,~(\ref{eq: stability for siblings}) captures the case of students that have at least one sibling. In this case, the same constraint as in~(\ref{eq: standard formulation for student optimal stable matching}) applies. \nacho{Remember to update the objective accordingly to force siblings assigned together in case of multiple optimal solutions.}
            % \begin{subequations}\label{eq: independent formulation with siblings}
            % \begin{alignat}{2}
            %   \min_{\mathbf{x}\in \cP, \mathbf{y} \in \cM(\mathbf{x})} \quad & \quad \sum_{(s,c)\in \cV} r_{s,c}\cdot x_{s,c}  \\
            %   st. \quad & \quad q_{c,g(s)} \lrp{1- \sum_{c'\succeq_s c} x_{s,c'}} \leq \sum_{\substack{s'\in \cS^{g(s)}:\\ s'\succ_c s}} x_{s', c}
            %   + \sum_{\substack{(s',s'') \in \cS\times \cS^{g(s)} :\\ s'\succ_c s,\; s''\prec_c s }}  y_{s',s'',c}\; ,\; \forall (s,c) \in \cV, \label{eq: independent stability for no siblings} \\
            %   & \quad x_{s,c} + \lrp{1-\sum_{c'\succeq_{s'} c} x_{s',c'} } \leq 2-\sum_{\substack{k\in f(s''): \\ k \succ_c s''}} y_{k, s'',c}, \; \forall s''\in \cS, s,s' \in f, f\in \cF \text{ with } s\succ_c s'\succ_c s'', c\in \cC \label{eq: independent stability for siblings} \\
            %   & \quad x_{s,c} + \lrp{1-\sum_{c'\succeq_{s'} c} x_{s',c'} } \leq 2 - x_{s'',c} + \sum_{\substack{k\in f(s''): \\ k \succ_c s''}} y_{k, s'',c}, \; \forall s''\in \cS, s,s' \in f, f\in \cF \text{ with } s\succ_c s'', c\in \cC \label{eq: independent stability for siblings v2}
            % \end{alignat}
            % \end{subequations}
            {\small
            \begin{subequations}\label{eq: independent formulation with siblings}
            \begin{alignat}{2}
              \min_{\mathbf{x}\in \cP, \mathbf{y} \in \cM(\mathbf{x})} \quad & \quad \sum_{(s,c)\in \cV} r_{s,c}\cdot x_{s,c}  \\
              st. \quad & \quad q_{c,g(s)} \lrp{1- \sum_{c'\succeq_s c} x_{s,c'}} \leq \sum_{\substack{s'\in \cS^{g(s)}:\\ s'\succ_c s}} x_{s', c}
              + \sum_{\substack{(s',s'') \in \cS^{g(s)\times f(s')} :\\ s' \prec_c s }}  y_{s'',s',c}\; ,\; \forall (s,c) \in \cV, \label{eq: constraint stability independent} \\
              % & \quad x_{s,c} + \lrp{1-\sum_{c'\succeq_{s'} c} x_{s',c'} } \leq 2-\sum_{\substack{k\in f(s''): \\ k \succ_c s''}} y_{k, s'',c}, \; \forall s''\in \cS, s,s' \in f, f\in \cF \text{ with } s\succ_c s'\succ_c s'', c\in \cC \label{eq: independent stability for siblings} \\
              & \quad \lrp{1-\sum_{c'\succeq_{s} c} x_{s,c'} } + x_{s',c} \leq 2 - x_{l,c}
              + \sum_{k\in f(l)} y_{k, l,c}\cdot \ind{l \succ_c s}, \; \forall c\in \cC, \lrl{s,s'} \subseteq f\in \cF, l\in \cS^{g(s)} \label{eq: tie breaking dynamic independent} \\
              & \quad \lrp{1-\sum_{c'\succeq_{s} c} x_{s,c'} } \leq 1 - x_{l,c}\cdot \ind{s \succ_c l}
              + \sum_{k\in f(l)} y_{k, l,c}\cdot \ind{s \succ_c l}, \; \forall c\in \cC, s, l\in \cS^{g(s)} \label{eq: respect priority independent}
            \end{alignat}
            \end{subequations}
            }

            The sets of constraints~(\ref{eq: constraint stability independent}) adapt the stability constraints~(\ref{eq: statibility for sosm}) to incorporate dynamic priorities and eliminate independent justified-envy. Specifically, a student \(s\) may not get assigned to school \(c\) (or better) if there are at least \(q_{c,g(s)}\) students with (i) higher priority assigned to \(c\) (first term on the right-hand side) or (ii) lower priority than \(s\) but that have a sibling assigned to the school, and thus obtain siblings' priority.
            The sets of constraints~(\ref{eq: tie breaking dynamic independent}) guarantee that ties among students with dynamic priority are broken according to their priorities, thus satisfying independent justified-envy. On the one hand, if student \(s\) has no sibling assigned to \(c\),
            then the left-hand side of~(\ref{eq: tie breaking among dynamic priorities for independent}) is at most one, and thus the constraint is always satisfied. On the other hand, if \(s\in f(s')\) is assigned to school \(c\), \(s'\) gets dynamic priority. As a result:
            \begin{itemize}
              \item No student \(l\in \cS^(g(s'))\) without dynamic priority can be assigned to \(c\) unless \(s'\) is assigned to \(c\) or better. This is captured by the fact that, if \(l\) has no dynamic priority, then the summation on the right-hand size is equal to zero, and thus \(x_{l,c} = 0\) if \(x_{s,c} = 1\) and \(\sum_{c'\succeq_s c} x_{s,c'} = 0\).
              \item Student \(s'\) may not get assigned to \(c\) or better only if student \(l\) has dynamic priority and is preferred by school \(c\).
            \end{itemize}
            Finally, the set of constraints~(\ref{eq: respect priority independent}) guarantees that priorities are respected among students with no dynamic priority. If student \(s\) is not assigned to \(c\) or better, the left-hand side is equal to 1, and thus a student less preferred \(l\) such that \(s \succ_c l\) can only be assigned to \(c\) if they have dynamic priority. On the other hand, if \(l\succ_c s\) then \(\ind{s\succ_c l} = 0\) and thus the constraint is always satisfied.


            In Theorem~\ref{thm: correctness of independent formulation based on BB} we show that the integer programming formulation~\ref{eq: independent formulation with siblings} solves Problem~\ref{problem_def} under \emph{independent justified-envy}.
            \begin{theorem}\label{thm: correctness of independent formulation based on BB}
              The formulation in~(\ref{eq: independent formulation with siblings}) solves Problem~\ref{problem_def} when we consider independent justified-envy.
            \end{theorem}

            \proof{Proof.}
                We need to show that
                \begin{enumerate}
                  \item The formulation returns an allocation that is stable with unilateral dynamic priorities and that is free of independent justified-envy.
                  \item The allocation obtained is student-optimal and maximizes the cardinality of the match.
                \end{enumerate}

                In a slight abuse of notation, we will refer \(x^{*}, y^{*}\) to the optimal solution of~(\ref{eq: independent formulation with siblings}), and we will denote by \(\mu^{*}\) the corresponding matching, i.e., \(\mu^*(s) = c\) if and only if \(x^*_{s,c} = 1\).
                Similarly, \(\mu^*(c) = \lrl{s\in \cS : x^*_{s,c} = 1}\), and we say that \((s,c) \in \mu^*\) if \(\mu^*(s) = c\) and \(s\in \mu^*(c)\), respectively.

                First, by definition of \(y\) we know that \(y_{s,s',c} = 1\) only if \(s\succ_c s'\) and both \(x_{s,c} = x_{s',c} = 1\). Hence, by definition of \(y\), it is direct that the allocation will satisfy that dynamic priorities are unilateral. Next, we show that the allocation that results from solving~(\ref{eq: independent formulation with siblings}) is free of independent justified-envy. To accomplish this we proceed by parts.

                \paragraph{Non-wastefulness.} To find a contradiction, suppose there exists \((s,c) \notin \mu^*\) such that \(c \succ_s \mu^*(s)\) and \(\lra{\mu^*(c) \cap \cS^{g(s)}} < q_{c,g(s)}\). Then, we know that
                \(\sum_{c'\succeq_s c} x^*_{s,c'} = 0\), and therefore the left-hand side of~(\ref{eq: constraint stability independent}) is equal to \(q_{c,g(s)}\). As a result, we must have that
                \[\sum_{\substack{s'\in \cS^{g(s)}:\\ s'\succ_c s}} x^*_{s', c} + \sum_{\substack{(s',s'') \in \cS^{g(s)\times f(s')} :\\ s' \prec_c s }}  y^*_{s'',s',c} \geq q_{c,g(s)}.\]
                Since there is no student \(s'\) satisfying that \(s' \succ_c s\) and \(s' \prec_c s\), we know that the two summations are over disjoint sets of students in \(\cS^{g(s)}\), and thus we know that
                \[\lra{\mu^*(c)} = \sum_{\substack{s'\in \cS^{g(s)}:\\ s'\succ_c s}} x^*_{s', c} + \sum_{\substack{(s',s'') \in \cS^{g(s)\times f(s')} :\\ s' \prec_c s }}  y^*_{s'',s',c} \geq q_{c,g(s)} \geq q_{c,g(s)}.\]
                Hence, we obtain a contradiction as we initially assumed that  \(\lra{\mu^*(c) \cap \cS^{g(s)}} < q_{c,g(s)}\).
                \paragraph{No independent justified-envy.} To find a contradiction, suppose there exists \((s,c) \notin \mu^*\) such that \(c \succ_s \mu^*(s)\) and that \(s\) has independent justified-envy towards student \(s' \in \cS^{g(s)}\). Then, there are two options:
                \begin{enumerate}
                  \item \(\lra{\zeta_{\mu^*} (s,c)} > \lra{\zeta_{\mu^*} (s',c)}\), in which case \(s\) has a sibling assigned to school \(c\) and \(s'\) does not. Hence, we have that \(x^*_{\tilde{s},c} = 1, \sum_{c'\succeq_s c} x_{s,c'} = 0\), and \(x_{s',c} = 1\), where \(\tilde{s} \in f(s)\). As a result, the left-hand side of~(\ref{eq: tie breaking dynamic independent}) is equal to 2 for \((\tilde{s}, s)\), and therefore \(-x^*_{s',c} + \sum_{k\in f(s')} y_{k,s',c}\cdot \ind{s' \succ s}\) must be equal to zero.
                  However, by hypothesis we know that \(s'\) has no siblings assigned to school \(c\), and thus the second term is equal to zero. Hence, for the constraint to be satisfied we need that \(x^*_{s',c} = 0\), which contradicts the initial hypothesis that student \(s'\) is assigned to school \(c\).
                  \item \(\lra{\zeta_{\mu^*} (s,c)} = \lra{\zeta_{\mu^*} (s',c)}\) and \(s\succ_c s'\), in which case both students have the same priority condition.
                  Since \(s\) has independent justified-envy towards \(s'\), we know that \(x_{s',c} = 1\), and \(\sum_{c'\succeq c} x_{s,c'} = 0\). Then, we have two sub-cases:
                  \begin{itemize}
                    \item If both students do not have dynamic priority, then the left-hand side of~(\ref{eq: respect priority independent}) is equal to 1. In addition, we know by hypothesis that \(s \succ_c s'\), and thus \(\ind{s\succ_c s'} = 1\). As a result, the right-hand side of~(\ref{eq: respect priority independent}) becomes \(1 - x_{s',c}
                    + \sum_{k\in f(s')} y_{k, s',c} = 1 - x_{s',c}\), where the latter follows since both \(s\) and \(s'\) have no dynamic priority in school \(c\). Finally, this implies that \(x_{s',c} = 0\), which contradicts our hypothesis that \(s\) has justified-envy.
                    \item If both students have dynamic priority, then~(\ref{eq: respect priority independent}) is always satisfied. On the other hand, the left-hand side of~(\ref{eq: tie breaking dynamic independent}) is equal to 2, and thus we must have that \(-x_{s',c}+ \sum_{k\in f(s')} y_{k, s',c}\cdot \ind{s' \succ_c s} = 0\).
                    Since \(\ind{s' \succ_c s} = 0\), we must have that \(x_{s',c} = 0\), which leads to a contradiction.
                  \end{itemize}
                \end{enumerate}
                \paragraph{Maximum caridnality.} To find a contradiction, suppose there exists another student-optimal assignment \(\mu'\) such that \(\sum_{s\in \cS} \ind{\mu'(s) \in \cC} > \sum_{s\in \cS} \ind{\mu^*(s) \in \cC}\). Let \(x'_{s,c} = 1\) if \(\mu'(s) = c\), and \(x'_{s,c} = 0\) otherwise, and let
                \(\mu'(\emptyset) = \lrl{s\in \cS: \mu'(s) = \emptyset}\) (similar definition applies for \(\mu^*(\emptyset)\)). Then,
                \[
                \sum_{(s,c)\in \mu'} r_{s,c} = \sum_{(s,c)\in \mu'} x'_{s,c}\cdot r_{s,c} + r_{\emptyset} \lra{\mu'(\emptyset)} < \sum_{(s,c)\in \mu'} x'_{s,c}\cdot r_{s,c} + r_{\emptyset} \lra{\mu^*(\emptyset)} = \sum_{(s,c)\in \mu^*} x^*_{s,c}\cdot r_{s,c} + r_{\emptyset} \lra{\mu^*(\emptyset)} = \sum_{(s,c) \in \cV} x^*_{s,c}\cdot r_{s,c},
                \]
                where the first equality follows by definition of \(x'\); the inequality follows by hypothesis (i.e., that the cardinality of \(\mu'\) is higher than that of \(\mu^*\)); the second equality follows by student-optimality of both assignments, and the last equality follows by definition of \(x^*\). As a result, we obtain that \(\sum_{s,c\in \cV} x'_{s,c}\cdot r_{s,c} < \sum_{(s,c) \in \cV} x^*_{s,c} \cdot r_{s,c}\), which contradicts the optimality of \(x^*\), leading to a contradiction.
            \Halmos\endproof



            % {\small
            % \begin{subequations}\label{eq: independent formulation with siblings}
            % \begin{alignat}{2}
            %   \min_{\mathbf{x}\in \cP, \mathbf{y} \in \cM(\mathbf{x})} \quad & \quad \sum_{(s,c)\in \cV} r_{s,c}\cdot x_{s,c}  \\
            %   st. \quad & \quad q_{c,g(s)} \lrp{1- \sum_{c'\succeq_s c} x_{s,c'}} \leq \sum_{\substack{s'\in \cS^{g(s)}:\\ s'\succ_c s}} x_{s', c}
            %   + \sum_{\substack{(s',s'') \in \cS^{g(s)\times f(s')} :\\ s' \prec_c s }}  y_{s'',s',c}\; ,\; \forall (s,c) \in \cV, \label{eq: constraint stability independent} \\
            %   % & \quad x_{s,c} + \lrp{1-\sum_{c'\succeq_{s'} c} x_{s',c'} } \leq 2-\sum_{\substack{k\in f(s''): \\ k \succ_c s''}} y_{k, s'',c}, \; \forall s''\in \cS, s,s' \in f, f\in \cF \text{ with } s\succ_c s'\succ_c s'', c\in \cC \label{eq: independent stability for siblings} \\
            %   & \quad \lrp{1-\sum_{c'\succeq_{s} c} x_{s,c'} } + x_{s',c} \leq 2 - x_{l,c}
            %   + \sum_{k\in f(l)} y_{k, l,c}\cdot \ind{l \succ_c s}, \; \forall c\in \cC, \lrl{s,s'} \subseteq f\in \cF, l\in \cS^{g(s)} \label{eq: tie breaking dynamic independent}
            %   & \quad \lrp{1-\sum_{c'\succeq_{s} c} x_{s,c'} } \leq 1 - x_{l,c}\cdot \ind{s \succ_c l}
            %   + \sum_{k\in f(l)} y_{k, l,c}\cdot \ind{s \succ_c l}, \; \forall c\in \cC, s, l\in \cS^{g(s)} \label{eq: respect priority independent} \\
            % \end{alignat}
            % \end{subequations}
            % }


        \subsubsection{Dependent case.}
            In the dependent case,
            % \begin{subequations}\label{eq: dependent formulation with siblings}
            % \begin{alignat}{2}
            %   \min_{\mathbf{x}\in \cP, \mathbf{y} \in \cM(\mathbf{x})} \quad & \quad \sum_{(s,c)\in \cV} r_{s,c}\cdot x_{s,c}  \\
            %   st. \quad & \quad q_{c,g(s)} \lrp{1- \sum_{c'\succeq_s c} x_{s,c'}} \leq \sum_{\substack{s'\in \cS^{g(s)}:\\ s'\succ_c s}} x_{s', c}
            %   + \sum_{\substack{(s',s'') \in \cS\times \cS^{g(s)} :\\ s'\succ_c s,\; s''\prec_c s }}  y_{s',s'',c}\; ,\; \forall (s,c) \in \cV, \label{eq: dependent stability for no siblings} \\
            %   & \quad x_{s,c} + \lrp{1-\sum_{c'\succeq_{s'} c} x_{s',c'} } \leq 2-\sum_{\substack{k\in f(l): \\ k \prec_c s}} y_{k, l,c}, \; \forall s,s' \in f, l\in \cS, c\in \cC \label{eq: dependent stability for siblings} \\
            %   & \quad x_{s,c} + \lrp{1-\sum_{c'\succeq_{s'} c} x_{s',c'} } \leq 2 - x_{l,c} + \sum_{\substack{k\in f(l): \\ k \succ_c s}} y_{k, l,c}, \; \forall s,s' \in f, l\in \cS, c\in \cC \label{eq: dependent stability for siblings v2}
            % \end{alignat}
            % \end{subequations}

            {\small
            \begin{subequations}\label{eq: dependent formulation with siblings}
            \begin{alignat}{2}
              \min_{\mathbf{x}\in \cP, \mathbf{y} \in \cM(\mathbf{x})} \quad & \quad \sum_{(s,c)\in \cV} r_{s,c}\cdot x_{s,c}  \\
              st. \quad & \quad q_{c,g(s)} \lrp{1- \sum_{c'\succeq_s c} x_{s,c'}} \leq \sum_{\substack{s'\in \cS^{g(s)}:\\ s'\succ_c s}} x_{s', c}
              + \sum_{\substack{(s',s'') \in \cS^{g(s)\times f(s') } :\\ s''\prec_c s }}  y_{s'',s',c}\; ,\; \forall (s,c) \in \cV, \label{eq: constraint stability dependent} \\
              & \quad x_{s,c} + \lrp{1-\sum_{c'\succeq_{s'} c} x_{s',c'} } \leq 2 - x_{l,c} + \sum_{k\in f(l)} y_{k,l,c}\cdot \ind{k \succ_c s}, \; \forall c\in \cC, \lrl{s,s'} \subseteq f\in \cF, l\in \cS^{g(s')} \label{eq: tie breaking dynamic dependent}\\
              & \quad \lrp{1-\sum_{c'\succeq_{s} c} x_{s,c'} } \leq 1 - x_{l,c}\cdot \ind{s \succ_c l}
              + \sum_{k\in f(l)} y_{k, l,c}\cdot \ind{s \succ_c l}, \; \forall c\in \cC, s, l\in \cS^{g(s)} \label{eq: respect priority dependent}
            \end{alignat}
            \end{subequations}}



            In Theorem~\ref{thm: correctness of dependent formulation based on BB} we show that the integer programming formulation~\ref{eq: dependent formulation with siblings} solves Problem~\ref{problem_def}.
            \begin{theorem}\label{thm: correctness of dependent formulation based on BB}
              The formulation in~(\ref{eq: dependent formulation with siblings}) solves Problem~\ref{problem_def} when we consider dependent justified-envy.
            \end{theorem}

            \proof{Proof.}
                First, note that the objective of Problems~\ref{eq: independent formulation with siblings} and~\ref{eq: dependent formulation with siblings} are the same. In addition, the sets of constraints~(\ref{eq: constraint stability dependent}) and~(\ref{eq: respect priority dependent}) are equivalent to~(\ref{eq: constraint stability independent}) and~(\ref{eq: respect priority independent}), respectively. Hence, the proofs for non-wastefulness and maximum cardinality are the same. Hence, we only need to show that the optimal solution of Problem~\ref{eq: dependent formulation with siblings} is free of dependent justified-envy.

                To show this, we proceed by contradiction. Suppose there exists \((s,c) \notin \mu^*\) such that \(c \succ_s \mu^*(s)\) and that \(s\) has dependent justified-envy towards student \(s' \in \cS^{g(s)}\). As for the independent case, there are two options:
                \begin{enumerate}
                  \item \(\zeta_{\mu^*} (s,c) \succ_c \zeta_{\mu^*} (s',c)\), in which case \(s\) has a sibling assigned to school \(c\) and either \(s'\) has no siblings assigned to \(c\) (i.e., \(\zeta_{\mu^*} (s',c) = \emptyset\)) or \(s'\) has a sibling assigned to \(c\) that has lower priority compared to \(\zeta_{\mu^*} (s,c)\). Then, we know that \(x_{\zeta_{\mu^*} (s,c),c} = 1\), and thus the left-hand side of~(\ref{eq: tie breaking dynamic dependent}) is equal to 2.
                  On the other hand, since either \(s'\) has no sibling or has a sibling with lower priority than \(\zeta_{\mu^*} (s,c)\), we know that \(\sum_{k\in f(l)} y_{k,s',c}\cdot \ind{k \succ_c \zeta_{\mu^*} (s,c)} = 0\), and therefore the right-hand side is equal to \(2-x_{s',c}\). As a result, we must have that \(x_{s',c} = 0\), which contradicts the hypothesis that \(s\) has dependent-justified envy towards \(s'\) in school \(c\).
                  \item \(\zeta_{\mu^*} (s,c) \sim_c \zeta_{\mu^*} (s',c)\) and \(s\succ_c s'\), in which case both students have no dynamic priority.
                  Since \(s\) has independent justified-envy towards \(s'\), we know that \(x_{s',c} = 1\), and \(\sum_{c'\succeq c} x_{s,c'} = 0\). Then, the left-hand side of~(\ref{eq: respect priority dependent}) is equal to 1, while the right-hand side is equal to \(1 - x_{s',c}\). Hence, we must have that \(x_{s',c} = 0\), which leads to a contradiction since we assumed that \(s\) has dependent justified-envy towards \(s'\) in school \(c\).
                \end{enumerate}
            \Halmos\endproof


    % \subsection{Cutoff formulation}
    %     Although in some systems students may not have a score, we can use schools' priorities to construct one.\footnote{Alternatively, we can use the random tie-breaker as a score. In general, since we only care about the order, any set of scores that preserve the order defined by \(\succ_c\) works.} Specifically, if student \(s\) has the \(k\)-th highest priority in school \(c\), then we assume their score is \(p_{s,c} = \frac{\lra{S^g_c}-{k} + 1}{\lra{S^g_c}}\). For example, the student \(s\) with ex-ante highest priority in school \(s\) has score \(p_{s,c} = 1\), while the ex-ante lowest priority student \(s'\) in that school has score \(p_{s',c} = \frac{1}{\lra{S^g_c}}\).
    %
    %     Based on these scores, an alternative approach to capture dynamic priorities is to extend the cutoff formulation described in~\cite{agoston2016integer}. This formulation, which relies on the notion of ascending score limits introduced in~\cite{biro2015college}, replaces the stability constraints~(\ref{eq: statibility for sosm}) by modeling the cutoff \(z_c\) of each school \(c\in \cC\), which is equal to the lowest score that guarantees admission to that school. To capture our setting, we extend this formulation by considering two cutoffs per school and grade level, \(w^g_c\) and \(z^g_{c}\). The first set
    %     of cutoffs, \(\lrl{w_{c}^g}_{c\in \cC, g\in \cG}\), captures the minimum score required to be granted admission in school \(c\) in grade \(g\) without dynamic priorities, i.e., for students with no siblings assigned to school \(c\). The second set of cutoffs, \(\lrl{z_{c}^g}_{c\in \cC, g\in \cG}\), represents the minimum score required to get admitted to school \(c\) in grade \(g\) if the student gets siblings' priority. By construction, note that \(z_c^g \leq w_c^g\) for all \((c,g)\in \cC\times \cG\). Moreover, let
    %     \begin{equation}\label{eq: set of score limits}
    %       \cL = \lrl{\mathbf{w}, \mathbf{z} \in [0,1]^{\cC\times \cG} \times [0,1]^{\cC\times \cG}\;:\; w_{c,g} \text{ and } z_{c,g} \text{ are cutoffs for } (c,g) \in \cC\times \cG}
    %     \end{equation}
    %     be the set of cutoffs for prioritized and non-prioritized students. Given this set, in Theorem~\ref{thm: correctness of formulation based on mincut} we show that Problem~\ref{problem_def} can be formulated as the following mixed-integer program:
    %     \begin{subequations}\label{eq: formulation based on mincut}
    %     \begin{alignat}{2}
    %       \min_{\substack{\mathbf{x} \in \cP, \mathbf{y}\in \cM(\mathbf{x}) \\ (\mathbf{w}, \mathbf{z}) \in \cL}} \quad & \quad \sum_{(s,c) \in \cV} x_{s,c}\cdot r_{s,c} \\
    %       st. \quad & \quad w_{c}^{g(s)} \leq (1-x_{s,c})\cdot \lrp{\bar{p}+1} + p_{s,c}\cdot \lrp{1+\frac{y_{s,s',c}}{p_{s,c}}},\; \forall (s,c) \in \cV, \; s'\in f(s) \label{eq: constraint_mincut_1} \\
    %       & \quad z_{c}^{g(s)} \leq (1-y_{s,s',c})\cdot \lrp{\bar{p}+1} + p_{s,c},\; \forall (s,c) \in \cV, s'\in f(s), \; \lra{f(s)} > 1  \label{eq: constraint_mincut_2} \\
    %       & \quad p_{s,c}+\epsilon \leq w_{c}^{g(s)} + \lrp{\sum_{c' \succeq_s c}x_{s,c'}}\lrp{\bar{p}+1},\; \forall (s,c) \in \cV, \; \lra{f(s)} = 1  \label{eq: constraint_mincut_3} \\
    %       & \quad p_{s,c}+\epsilon \leq z_{c}^{g(s)} + \lrp{y_{s,s',c} + \sum_{c' \succ_s c} x_{s,c'}}\lrp{\bar{p}+1},\; \forall  (s,c) \in \cV, \; s'\in f(s), \; \lra{f(s)} > 1  \label{eq: constraint_mincut_4}
    %     \end{alignat}
    %     \end{subequations}
    %
    %     \nacho{Confirm that this extension works. In any case, the implementation with only one grade is commented right below.}
    %     Constraints~(\ref{eq: constraint_mincut_1}),~(\ref{eq: constraint_mincut_2}),~(\ref{eq: constraint_mincut_3}) and~(\ref{eq: constraint_mincut_4}) replace the stability constraints in~(\ref{eq: standard formulation with siblings}) by properly defining the cutoffs for prioritized and non-prioritized students. More specifically, constraints~(\ref{eq: constraint_mincut_1}) guarantees that student \(s\) is admitted to school \(c\) if their score \(p_{s,c}\) is above the cutoff \(w_{c}^{g(s)}\), unless \(s\) has a siblings \(s'\) assigned to the same school. In the latter case, the constraint will not be binding, and thus student \(s\) may be admitted with a lower score. Constraint~(\ref{eq: constraint_mincut_2}) enforces that, if student \(s\) is assigned to \(c\) along with their sibling \(s'\), then their score is higher than the cutoff for prioritized students. Finally, constraints~(\ref{eq: constraint_mincut_3}) and~(\ref{eq: constraint_mincut_4}) ensure that if student \(s\) is not assigned to \(c\) or to another school they prefer, then their score is below the cutoff of school \(c\).
    %
    %
    %     \begin{theorem}\label{thm: correctness of formulation based on mincut}
    %       The formulation in~(\ref{eq: formulation based on mincut}) solves Problem~\ref{problem_def}.
    %     \end{theorem}


%!TEX root = ./0_main.tex
\section{Heuristics}\label{sec: heuristics}

    Given the complexity of the problem, solving the formulations described in Section~\ref{sec: exact} may take a significant amount of time.
    In this section, we introduce several heuristics that can reduce computational times. Specifically, in Section~\ref{subsec: preprocessing} we discuss preprocessing heuristics that can reduce the size of the problem while keeping the set of optimal solutions. For cases when this preprocessing is not enough to solve the problem reasonably fast, in Section~\ref{subsec: near optimal heuristics} we propose heuristics to find near-optimal solutions quickly.

    \subsection{Preprocessing}\label{subsec: preprocessing}

        \nacho{Not sure if these are definitions of propositions. I guess propositions because we need to show that they cannot belong to any stable matching.}
        \begin{proposition}
          A pair \((s,c)\) is school-dominated if there exists \(c' \succ_s c\)
          \[\lra{\lrl{s' \in \cS\setminus \lrl{s} \;:\; s'\succ_{c} s}} \leq q_{c,g(s)}\]
        \end{proposition}

        \begin{proposition}
          A pair \((s,c)\) is student-dominated if
          \[
          \lra{\lrl{s'\in \cS\setminus \lrl{s} \;:\; s' \succ_{c} s \text{ and } c\succ_{s'} c' \; \forall c'\in \cC, \; }} >  q_{c,g(s)}
          \]
        \end{proposition}


        \begin{proposition}\label{prop: school-domination preserved}
          If a pair \((s,c)\) is school-dominated in the case with no dynamic priorities, then it is also school-dominated in the case with dynamic priorities.
        \end{proposition}

        Proposition~\ref{prop: school-domination preserved} implies that school-dominated nodes will remain dominated after dynamic priorities are taken into account. However, as the following example illustrates, this may not be the case when we consider student-dominated pairs.

        \begin{example}
          Consider an instance with two levels, \(G = \lrl{g_1, g_2}\), one family \(f = \lrl{f_1, f_2}\), and two additional students, \(\lrl{a_1, a_2}\). Students' subscripts indicate the level to which they belong, i.e., \(\cS^{g_1} = \lrl{f_1, a_1}\) and \(\cS^{g_2} = \lrl{f_2, a_2}\). In addition, suppose that there are two schools \(\cC = \lrl{c_1, c_2}\), each offering one seat per level, and suppose that every student prefers \(c_1\) over \(c_2\). Finally, suppose that \(f_1 \succ_c a_1\) and \(a_2 \succ_c f_2\) for every school \(c\in \cC\).

          In this case, if dynamic priorities did not exist, then the pair \((f_2, c_1)\) would be student-dominated by the pair \((a_2, c_1)\), since student \(a_2\)'s top preference is \(c_1\) and \(c_1\) prefers \(a_2\) over \(f_2\). However, with dynamic priorities, we know that student \(f_1\) will be assigned to school \(c_1\), and thus \(f_2\) will get dynamic priority and will no longer be student-dominated.
        \end{example}

        \begin{proposition}
          If a pair \((s,c)\) is student-dominated under dynamic priorities if
          \[
          \lra{\lrl{s'\in \cS\setminus \lrl{s} \;:\; s' \succ_{c} s \text{ and } c\succ_{s'} c' \; \forall c'\in \cC, \; }} > q_{c,g(s)}
          \]
        \end{proposition}



        \begin{algorithm}
          bla
        \end{algorithm}

    \subsection{Near-Optimal Approaches}\label{subsec: near optimal heuristics}

        We consider three families of heuristics:
        \begin{itemize}
          \item Sequential DA~\citep{Correa_2022}: This heuristic first defines an order among grade levels (e.g., decreasing from 12th grade to Pre-K), and then sequentially solves the assignment for each grade level using DA. Before moving to the next level, the algorithm updates priorities given by the assignments in the levels already processed, and continues until processing the last level.
          \item DA and re-allocation: This heuristic starts solving the assignment problem without considering any dynamic priorities. Then, the algorithm identifies students whose siblings were assigned to schools they prefer more, and assigns them to those schools unless they filled their quota with other families. After this, the algorithm computes the number of seats left in each school and level, and then computes the match on the residual graph that involves the remaining capacities and the students who were not previously assigned.
          \item NRMP algorithm: This algorithm works when participants submit joint lists, and it proceeds same as DA but couples are processed together and thus take two spots every time they propose. If one of the members of the couple is rejected, then both are rejected and they must move to their next joint preference. Note that this algorithm will not perform appropriately in our context because families do not report joint preferences. As a result, list are short, and thus families will most likely not get assigned.
          \item
        \end{itemize}


%!TEX root = ./0_main.tex
\section{Experiments} \label{sec: experiments}

  \subsection{Synthetic Data}

      \subsubsection{Setting and Data.}

      \subsubsection{Results.}

  \subsection{Real Data: School Choice in Chile}

      \subsubsection{Setting and Data.}

      \subsubsection{Results.}


%!TEX root = ./0_main.tex
\section{Conclusions} \label{sec: conclusions}


\bibliographystyle{style/informs2014}
% {\tiny \bibliography{draft.bib}} % if more than one, comma separated
\bibliography{bibliography.bib} % if more than one, comma separated

\ACKNOWLEDGMENT{Potential feedback: Peng Shi, Nick Arnosti, Faidra Monachou, Nikhil Garg, Itai Ashlagi, Alf + Polytechnique Montreal, Rakesh Vohra}







\clearpage
%!TEX root = ./0_main.tex
%!TEX spellcheck = en-US
\begin{APPENDICES}

\section{Additional Results}\label{app: additional results}

    \begin{theorem}\label{thm: no stable matching}
      A stable matching may not exist under dynamic preferences, even in the case where families independent.
    \end{theorem}
    \begin{proof}{Proof.}
      Consider an instance with two families, \(f = \lrl{f_1,f_2}\) and \(f'=\lrl{f_1', f_2'}\), two schools \(\lrl{c_1, c_2}\) with two grades \(G = \lrl{g_1, g_2}\), and suppose that each school offers one seat in each grade. We assume that \(g_1\) is the lower grade, and \(g_2\) the higher grade, and we assume that the subindex in the members of each family represents the grade to which they are applying. Finally, the preferences and initial priorities are:
      \begin{equation}
        \begin{aligned}
          \succ_i & : c_1 \succ_i c_2, \quad \forall i\in f\cup f' \\
          \succ_{c_1}^{g_1}&: f_1 \succ_{c_1}^{g_1} f_1' \\
          \succ_{c_1}^{g_2}&: f_2' \succ_{c_1}^{g_2} f_2 \\
          \succ_{c_2}^g &: f_g' \succ_{c_2}^g f_g, \quad \forall g \in \lrl{g_1, g_2}
        \end{aligned}
      \end{equation}
      There are two possible allocations:
      \begin{itemize}
        \item if \(\mu(f_1) = c_1\), then \(f_2\) gets dynamic priority, and thus \(\mu(f_2) = c_1\). However, student \(f_2'\) has justified envy in school \((c_2)\) given the initial preferences.
        \item if \(\mu(f_2') = c_1\), then \(f_1'\) gets dynamic priority, and thus \(\mu(f_1') = c_1\). However, student \(f_1\) has justified envy in school \((c_2)\) given the initial preferences.
      \end{itemize}
      As a result, we observe that there is no envy-free stable assignment.
    \end{proof}

% \section{Quotas}\label{sec: quotas}
%
%     \begin{proposition}
%       Maximum compliance with reservations does not guarantee maximum cardinality of the match under ex-post stability.
%     \end{proposition}
%
%     \begin{proof}{Proof.}
%       Consider a market with \(I = \lrl{i_1, \ldots, i_4}\), \(J = \lrl{c_1, c_2}\), \(T=\lrl{d,h}\), \(q_{c_1}=3\), \(q_{c_1,h} = q_{c_1,d} = 1\), \(q_{c_2} = 1\), and \(q_{c_2, t} = 0\).
%       In addition, suppose that \(\tau(i_1) = \emptyset, \; \tau(i_2) = \lrl{h,d}, \; \tau(i_3) = \lrl{h}, \; \tau(i_4) = \lrl{d}\), that all students prefer \(c_1\) over \(c_2\), and that schools priorities are:
%       \begin{equation}
%         \begin{aligned}
%           \succ_{c_1, \emptyset}&: i_1, i_2, i_3, i_4 \\
%           \succ_{c_1, h}&: i_2, i_3, i_1, i_4 \\
%           \succ_{c_1, d}&: i_2, i_4, i_1, i_3 \\
%           \succ_{c_2, \emptyset}&: i_1, i_2, i_3, i_4 \\
%         \end{aligned}
%       \end{equation}
%       Then, following the horizontal envelope choice rule we obtain the assignment \(\mu = \lrl{(i_1, s_1), (i_2, c_1), (i_3, c_1), (i_4, \emptyset)}\). In contrast, using the Chilean mechanism (which assumes that all students prefer \(d\) over \(h\) and then open seats at any given school), we obtain the assignment \(\mu' = \lrl{(i_1, s_1), (i_2, c_1), (i_3, c_2), (i_4, c_1)}\), and thus we obtain an assignment that is ex-post stable and matches strictly more students than the assignment satisfying maximum compliance with reservations.
%     \end{proof}
%
%
%     \begin{problem}\label{problem_def with quotas}
%     Given an instance $\Gamma$, the \emph{maximum stable-matching with quotas} is the following:
%     \begin{equation*}
%     \min_{\mu} \Bigg\{ \sum_{j\in J} \lra{\mu(j)} \; : \;  \mu \text{ is a ex-post student-optimal stable matching }
%      \Bigg\}.
%     \end{equation*}
%     \end{problem}
%
%     \begin{claim}
%       The problem of finding the \emph{maximum stable-matching with quotas} is NP-hard.
%     \end{claim}
%
%
%     \subsection{Formulation}
%
%         \begin{equation}\label{eq: formulation of problem with quotas}
%           \begin{aligned}
%             \min \quad & \quad \sum_{(i.j) \in I\times J, t\in T} x_{i,j,t}\cdot r_{i,j} + [\text{penalty for unassigned}] + [\text{penalty for traits}] \\
%             st. \quad & \quad \sum_{j\in J, t\in T} x_{i,j,t} \leq 1, \; \forall i\in I \\
%             & \quad \sum_{i\in I} x_{i,j,t} \leq q_{c,t}, \; \forall j\in J, \; t\in T \\
%             & \quad z_{j,t} \leq (1-x_{i,j,t})\cdot \lrp{\bar{s}+1} + s_{i.j,t},\; \forall i\in I,; j\in J,\; t\in T \\
%             & \quad s_{ijt}+\epsilon \leq z_{j,t} + \lrp{\sum_{k \succeq_i j, \; t\in T}x_{i,k,t}}\lrp{\bar{s}+1},\; \forall i\in I,; j\in J,\; t\in T \\
%             & \quad x_{i,j,t}\in \lrl{0,1}, \; i\in I, \; j\in J, \; t\in T \\
%             & \quad z_{j,t} \in [0,1], \; j\in J, \; t\in T \\
%           \end{aligned}
%         \end{equation}
%
%
%         \begin{theorem}
%           The formulation in~(\ref{eq: formulation of problem with quotas}) solves Problem~\ref{problem_def with quotas}.
%         \end{theorem}



\end{APPENDICES}











\end{document}
